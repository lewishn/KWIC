\documentclass[12pt,a4paper]{report}
\usepackage[utf8]{inputenc}
\usepackage[english]{babel}
\usepackage{amsmath}
\usepackage{amsfonts}
\usepackage{amssymb}
\usepackage{hyperref}
\usepackage{tikz}
\usetikzlibrary{shapes,arrows}
\usetikzlibrary{positioning}
\usepackage{graphicx}
\usepackage[left=2cm,right=2cm,top=2cm,bottom=2cm]{geometry}
\begin{document}

\tikzstyle{block} = [rectangle, draw, fill=blue!20, 
    text width=7em, text centered, rounded corners, minimum height=4em]
\tikzstyle{line} = [draw, -latex']
\tikzstyle{cloud} = [draw, ellipse,fill=red!20, node distance=3cm,
    minimum height=2em]


\chapter*{Assignement 1: KWIC-KWAC-KWOC}

Code Repository URL: \url{https://github.com/lewishn/KWIC.git} 

\begin{center}
\begin{tabular}{|c|c|c|}
\hline 
Name & Meike Aichele & Lewis Haris Nata\\ 
\hline 
Matriculation Nr. & A0128157A & A.... \\ 
\hline 
\end{tabular} 
\end{center}

\section*{1. Introduction}

\section*{2. Design}
The design of the program is as followed: There is a main class called TextProcessor which waits for input and calls the Action class that the user enters. All classes that should be executed by TextProcessor have to inherit the Action-Interface that provides a method called execute(). For the task in this assignment we provided a class KWIC that inherits the Action interface and implements the execute-Method.
The KWIC class itself contains the execute-Method, a method getKWIC which first rotates the String array then sorts it alphabetically and then saves the results into a text file.

We also implemented a simple text file reader and writer that can be used for extended classes as well. The class diagram of our program is shown in figure \ref{fig1}.

\begin{figure}[h]
\centering
\begin{tikzpicture}[node distance = 5cm, auto]
% draw nodes
\node [block, text width=10em,rectangle split, rectangle split parts=2] (main) {TextProcessor 
\nodepart{second}main(); \\ runAction(String);};
\node [block,rectangle split, rectangle split parts=2, right of=main] (interface) {Action
\nodepart{second}execute();};
\node [block, text width=10em,rectangle split, rectangle split parts=2, below of=interface] (kwic) [below =-3cm] {KWIC
\nodepart{second}execute();\\ getKWIC();\\ stringRotate(String); \\ ...};
\node [block,rectangle split, rectangle split parts=2, right of=kwic] (other) {...
\nodepart{second}execute();\\ ...};
\node[draw=white,right of=other](dots)[right=-1.6cm]{...};
%draw arrows
\path [line,dashed] (kwic) -- (interface);
\path [line,dashed] (other) -- (interface);
\path [line,dashed] (dots) -- (interface);
\path [line] (main) -- (interface);
\end{tikzpicture}
\caption{Diagram of the program structure.}
\label{fig1}
\end{figure}

\section*{3. Limitations and Benefits}
\end{document}